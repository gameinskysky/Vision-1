\chapter*{Abstract}
\addcontentsline{toc}{chapter}{Abstract}

Monocular visual-inertial localization has become more and more popular in autonomous mobile robotics. Combining the short-time accuracy of an inertial measurement unit (IMU) with the rich structure information of a camera, accurate localization can be achieved. ROVIO, robust visual-inertial odometry, and OKVIS, optimal keyframe-based visual-inertial SLAM, are two promising algorithms for visual-inertial localization. This work presents a direct comparison between the two algorithms. The comparison emphasizes drift issues, especially in the orientation around the world-z axis. An in-depth analysis on global and local accuracy and a proper definition of accuracy measures is presented. So far, ROVIO and OKVIS have only been proven to run on a hardware-wise time synchronized sensor system. We present experimental results on ROVIO running on a more generic, non time synchronized, sensor system and show its principle ability to run without time synchronization.










